\documentclass[journal,12pt,twocolumn]{IEEEtran}

\usepackage{setspace}
\usepackage{gensymb}
\singlespacing
\usepackage[cmex10]{amsmath}

\usepackage{amsthm}

\usepackage{mathrsfs}
\usepackage{txfonts}
\usepackage{stfloats}
\usepackage{bm}
\usepackage{cite}
\usepackage{cases}
\usepackage{subfig}

\usepackage{longtable}
\usepackage{multirow}

\usepackage{enumitem}
\usepackage{mathtools}
\usepackage{steinmetz}
\usepackage{tikz}
\usepackage{circuitikz}
\usepackage{verbatim}
\usepackage{tfrupee}
\usepackage[breaklinks=true]{hyperref}
\usepackage{graphicx}
\usepackage{tkz-euclide}

\usetikzlibrary{calc,math}
\usepackage{listings}
    \usepackage{color}                                            %%
    \usepackage{array}                                            %%
    \usepackage{longtable}                                        %%
    \usepackage{calc}                                             %%
    \usepackage{multirow}                                         %%
    \usepackage{hhline}                                           %%
    \usepackage{ifthen}                                           %%
    \usepackage{lscape}     
\usepackage{multicol}
\usepackage{chngcntr}

\DeclareMathOperator*{\Res}{Res}

\renewcommand\thesection{\arabic{section}}
\renewcommand\thesubsection{\thesection.\arabic{subsection}}
\renewcommand\thesubsubsection{\thesubsection.\arabic{subsubsection}}

\renewcommand\thesectiondis{\arabic{section}}
\renewcommand\thesubsectiondis{\thesectiondis.\arabic{subsection}}
\renewcommand\thesubsubsectiondis{\thesubsectiondis.\arabic{subsubsection}}


\hyphenation{op-tical net-works semi-conduc-tor}
\def\inputGnumericTable{}                                 %%

\lstset{
%language=C,
frame=single, 
breaklines=true,
columns=fullflexible
}
\begin{document}


\newtheorem{theorem}{Theorem}[section]
\newtheorem{problem}{Problem}
\newtheorem{proposition}{Proposition}[section]
\newtheorem{lemma}{Lemma}[section]
\newtheorem{corollary}[theorem]{Corollary}
\newtheorem{example}{Example}[section]
\newtheorem{definition}[problem]{Definition}

\newcommand{\BEQA}{\begin{eqnarray}}
\newcommand{\EEQA}{\end{eqnarray}}
\newcommand{\define}{\stackrel{\triangle}{=}}
\bibliographystyle{IEEEtran}
\raggedbottom
\setlength{\parindent}{0pt}
\providecommand{\mbf}{\mathbf}
\providecommand{\pr}[1]{\ensuremath{\Pr\left(#1\right)}}
\providecommand{\qfunc}[1]{\ensuremath{Q\left(#1\right)}}
\providecommand{\sbrak}[1]{\ensuremath{{}\left[#1\right]}}
\providecommand{\lsbrak}[1]{\ensuremath{{}\left[#1\right.}}
\providecommand{\rsbrak}[1]{\ensuremath{{}\left.#1\right]}}
\providecommand{\brak}[1]{\ensuremath{\left(#1\right)}}
\providecommand{\lbrak}[1]{\ensuremath{\left(#1\right.}}
\providecommand{\rbrak}[1]{\ensuremath{\left.#1\right)}}
\providecommand{\cbrak}[1]{\ensuremath{\left\{#1\right\}}}
\providecommand{\lcbrak}[1]{\ensuremath{\left\{#1\right.}}
\providecommand{\rcbrak}[1]{\ensuremath{\left.#1\right\}}}
\theoremstyle{remark}
\newtheorem{rem}{Remark}
\newcommand{\sgn}{\mathop{\mathrm{sgn}}}
%\providecommand{\abs}[1]{\left\vert#1\right\vert}
\providecommand{\res}[1]{\Res\displaylimits_{#1}} 
%\providecommand{\norm}[1]{\left\lVert#1\right\rVert}
%\providecommand{\norm}[1]{\lVert#1\rVert}
\providecommand{\mtx}[1]{\mathbf{#1}}
%\providecommand{\mean}[1]{E\left[ #1 \right]}
\providecommand{\fourier}{\overset{\mathcal{F}}{ \rightleftharpoons}}
%\providecommand{\hilbert}{\overset{\mathcal{H}}{ \rightleftharpoons}}
\providecommand{\system}{\overset{\mathcal{H}}{ \longleftrightarrow}}
	%\newcommand{\solution}[2]{\textbf{Solution:}{#1}}
\newcommand{\solution}{\noindent \textbf{Solution: }}
\newcommand{\cosec}{\,\text{cosec}\,}
\providecommand{\dec}[2]{\ensuremath{\overset{#1}{\underset{#2}{\gtrless}}}}
\newcommand{\myvec}[1]{\ensuremath{\begin{pmatrix}#1\end{pmatrix}}}
\newcommand{\mydet}[1]{\ensuremath{\begin{vmatrix}#1\end{vmatrix}}}
\numberwithin{equation}{subsection}
\makeatletter
\@addtoreset{figure}{problem}
\makeatother
\let\StandardTheFigure\thefigure
\let\vec\mathbf
\renewcommand{\thefigure}{\theproblem}
\def\putbox#1#2#3{\makebox[0in][l]{\makebox[#1][l]{}\raisebox{\baselineskip}[0in][0in]{\raisebox{#2}[0in][0in]{#3}}}}
     \def\rightbox#1{\makebox[0in][r]{#1}}
     \def\centbox#1{\makebox[0in]{#1}}
     \def\topbox#1{\raisebox{-\baselineskip}[0in][0in]{#1}}
     \def\midbox#1{\raisebox{-0.5\baselineskip}[0in][0in]{#1}}
\vspace{3cm}
\title{Assignment 1}
\author{Ch Pranay Prakash - EE18BTECH11009}
\maketitle
\newpage
\bigskip
\renewcommand{\thefigure}{\theenumi}
\renewcommand{\thetable}{\theenumi}
Download all C codes from 
\begin{lstlisting}
https://github.com/pranayEE11009/C_and_DataStructures/tree/main/Assignment_1/codes
\end{lstlisting}
%
and latex-tikz codes from 
%
\begin{lstlisting}
https://github.com/pranayEE11009/C_and_DataStructures/tree/main/Assignment_1
\end{lstlisting}
\section{Problem}
Consider the following ANSI C function:
\begin{lstlisting}
int SimpleFunction(int Y[], int n, int x)
{
int total = Y[0], loopIndex;
for (loopIndex=1; loopIndex<=n-1; loopIndex++){
    total=x*total +Y[loopIndex];
    }
return total;
} 
\end{lstlisting}
Let Z be an array of 10 elements with Z[i]=1, for all i such that 0$\le$i$\le$9. The value returned by SimpleFunction(Z,10,2) is ?
\section{Solution}
{\textbf {Solution: 1023}}\\

Code to generate the solution:
\begin{lstlisting}
#include <stdio.h>
int SimpleFunction(int Y[], int n, int x){
    int total = Y[0], loopIndex;
    for( loopIndex = 1; loopIndex<=n-1; loopIndex++){
        total = x*total + Y[loopIndex];
    }return total;
}
int main()
{
    int Z[10] = {1,1,1,1,1,1,1,1,1,1};
    printf("%d", SimpleFunction(Z, 10, 2));
    return 0;
}
\end{lstlisting}

The function {\textbf SimpleFunction} of the C code in the question takes an integer type array (Y[]), and two integer variables (n and x) as the inputs and returns an integer as the output.\\

The inputs of the SimpleFunction are:
\begin{enumerate}
    \item integer type array, Z[i] = 1 for all 0$\le$i$\le$9 \\
    i,e., Z = [1,1,1,1,1,1,1,1,1,1] 
    \item integer n = 10
    \item integer x = 2
\end{enumerate}

In the function SimpleFunction(Z,n,x) a "for loop" is run for n-1 iterations. \\
In each iteration the integer variable  "\textbf{total}", which is initiated with 1, is recursively multiplied with 2 with its previous value and added to 1.\\
\begin{lstlisting}
    total=x*total + Z[loopIndex];
\end{lstlisting}
Since, Z[i] is always 1 and $x = 2$.\\
\begin{lstlisting}
    total=2*total + 1;
\end{lstlisting}

The values of total for n-1 iterations are,\\
initially total = 1\\
for loopIndex = 1, total = 2*(1) + 1 = 3\\
for loopIndex = 2, total = 2*(3) + 1 = 7\\
for loopIndex = 3, total = 2*(7) + 1 = 15\\
for loopIndex = 4, total = 2*(15) + 1 = 31\\
for loopIndex = 5, total = 2*(31) + 1 = 63\\
for loopIndex = 6, total = 2*(63) + 1 = 127\\
for loopIndex = 7, total = 2*(127) + 1 = 255\\
for loopIndex = 8, total = 2*(255) + 1 = 511\\
for loopIndex = 9, total = 2*(511) + 1 = 1023\\

The for loop terminates at loopIndex = 9, and the SimpleFunction returns the final value of total, which is equal to \textbf{1023}.\\

Now, as we observe the values of "total" (3,7,15...,1023), we can observe that each value of "total" is one less than some integer exponential of 2. For example; \\ 

for loopIndex = 1, total = $3 = 2^{2} - 1$ \\
for loopIndex = 2, total = $7 = 2^{3} - 1$ and so on.\\

So, lets take a general equation of total, \\ 
\begin{lstlisting}
    T(m) = 2T(m-1) + 1
\end{lstlisting}
{\emph {where m is the iterative loopIndex of the for loop}}\\

Now,
\begin{gather*}
    T(m) = 2T(m-1) + 1 \\
    T(m-1) = 2T(m-2) + 1 \\
    T(m-2) = 2T(m-3) + 1 \\
    \vdots\\
    T(2) = 2T(1) + 1   
\end{gather*}
$\because loopIndex\ge1, m\ge1$

Now, to get a general solution to the above equations we multiply each equation with suitable coefficients and add them, \\
\begin{gather*}
    T(m) = 2T(m-1) + 1 \\
    2T(m-1) = 2*(2T(m-2) + 1) \\
    2^{2}*T(m-2) = 2^{2}*(2T(m-3) + 1) \\
    \vdots\\
    2^{m-2}*T(2) = 2^{m-2}*(2T(1) + 1) 
\end{gather*}

Now, on adding all the above equations we get rid of all the T(m-i) form values except T(m), T(1) and left with all the 1's from each equation,
\begin{gather*}
    T(m) = 1 + 2 + 2^{2} + 2^{3} + ... + 2^{m-2} + {2^{m-1}}T(1) \\ 
    T(m) = 2^{m-1} - 1 + 2^{m-1}*3 \\
    [\because T(1) = 2*(1) + 1 = 3] \\ 
    T(m) = 4*2^{m-1} - 1 \\
    T(m) = 2^{m+1} - 1
\end{gather*}

We can find "total" value using the equation below, where m is the iterative index "loopIndex".
\begin{gather}
    T(m) = 2^{m+1} - 1
\end{gather}

Since the SimpleFunction returns the total value, which corresponds for loopIndex = n-1.\\
So, for m = n-1, we have \\
\begin{gather}
    T(n-1) = 2^{n} - 1
\end{gather}

Finally, we can find the final output of the SimpleFunction using the above equation for different values of n.\\

In the question, the value of n=10,
\begin{gather*}
    \implies total = 2^{10} - 1 = \textbf{1023}
\end{gather*}
\\
\emph{\textbf{Note:} The above equation is only for x = 2 and Z = [1,1,1,1,1,1,1,1,1,1]}

\end{document}
